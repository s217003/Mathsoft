\documentclass{ctexart}

\usepackage{xltxtra}
\usepackage{pifont}



\title{作业三:对自己Linux工作环境的介绍}
\author{邵柯欣\\专业:信息与计算科学 学号:3200103310}

\begin{document}
\maketitle
~\\

\section{我的Linux发行版名称以及版本号}

\begin{itemize}
\item 发行版名称:Ubuntu
\item 版本号:Ubuntu 20.04.4 LTS
\end{itemize}
~\\


\section{我对系统做了哪些主要的调整,安装了哪些软件,做了哪些额外的配置工作}

\begin{itemize}
\item 通过变更\verb|/ect/apt/sources.list$|文件(当然,有保存原文件为\verb|/ect/apt/sources.list.old$|){\bf 更改了下载源}为阿里。
\item 用命令,安装$synaptic$用来安装软件。在synaptic中,下载$fcitx,fcitx-googlepinyin$
\item 在Language Support中选择fcitx,并{\bf 设置语言和键盘}。
  \newpage
\item 打开synaptic,{\bf 安装扩展包}:
\\$g++,gcc,$
\\$make,cmake,aotumake,$
\\$emacs,emacs-goodies-el$
\\$gedit,$
\\$texlive-full,$
\\$doxygen,doxygen-doc,$
\\$libboos-all-dev,$
\\$trllinos-all-dev,trllinos-dbg,trllinos-doc,$
\\$dx,$
\\$git,ssh,vnc4server,x11vnc,$
\\$virtualbox$
\item 在网上下载$doxymacs$安装包。
\item {\bf 配置工作}:
\\\ding{172}在\verb|.emacs|文件中配置doxymacs和emacs的其他参数
\\\ding{173}在\verb|.bashrc|文件中配置添加\verb|alias emacs='LC_CTYPE=zh_CN.utf8 emacs'|
\item 从网上安装钉钉、QQ、坚果云。
\end{itemize}
~\\


\section{规划我的下一步工作}

\subsection{预计未来半年我将在什么场合下使用Linux环境}

\begin{itemize}
\item 上课编程敲代码;
\item 写实验报告、论文;
\item 做课程笔记;
\item 也许打游戏(看不见我看不见我);
\item 。。。
\end{itemize}

\subsection{分析我目前的工作环境是否符合未来需求,如果不是,计划作出什么样的改变}

{\bf 我感觉我目前的工作环境已经可以满足我未来的需要了}\cite{PMSAJoP2018}。此处(随便)插入一个文献引用\verb|(>_<)|。

\begin{itemize}
\item 我对目前工作环境中的文件目录、编译器页面、指令快捷键等都比较熟悉,可以相对愉快的完成老师布置的任务和自己需要文件的编写。
\item 它稳定不崩盘(只要我不做死去探索新世界)。
\end{itemize}
~\\


\section{我将如何保证我的工作系统中的代码、文献和工作结果的稳定和安全}
\begin{itemize}
\item {\bf 关于代码}:我在{\bf gitee和github}上都有帐号,平时的代码都会及时上传。
\item {\bf 关于文献资料}:我使用{\bf 坚果云和百度网盘}(其他更加优秀可以存放大型文件的网盘还在寻找)来保存,小型的文件会放到坚果云里,大型的(一般不考虑重新下回来)会扔到百度网盘(下载速度感人)里。
\end{itemize}


\newpage
\bibliographystyle{plain}
\bibliography{ZL.bib}

\end{document}
