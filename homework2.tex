\documentclass{ctexart}
\usepackage{xltxtra}

\title{作业二:shell编程的返回值实现}

\author{邵柯欣\\专业:信息与计算科学 学号:3200103310}

\begin{document}

\maketitle
\pagestyle{empty}

\section{代码(分为两个部分)}
来源:Beginning+Linux+Programming的第48页
\subsection{第一部分:子函数}
~\\
\\创建\verb|yes_or_no|函数,用来根据用户的反馈确定返回值。
\begin{verbatim}
#!/bin/bash

yes_or_no(){
  echo "Is your name $* ?"     %打印带有用户输入行参的语句
  while true
  do
    echo -n "Enter yes or no:" %第一次提示用户选择
    read x                     %存储用户的输入
    case "$x" in               %根据用户的反馈选择输出
      y | yes ) return 0;;
      n | no )  return 0;;
      * )       echo "Answer yes or no"
      esac
    done
}
\end{verbatim}

\subsection{第二部分:main函数}
\begin{verbatim}
echo "Original parameter are $*"

if yes_or_no "$1"
then
  echo "Hi $1, nice name"
else
  echo "Never mind"
fi
exit 0
\end{verbatim}
~\\
~\\

\section{对文件权限的修改}
~\\  
\\创建一个\verb|My_name|文件,将上面两段代码写到该文件中。
~\\  
\\但是,如果你直接用
\\\verb|./My_name|
\\命令去尝试运行这个文件,你会得到
\\\verb|bash: ./My_name: 权限不足|
\\的错误反馈。
\\这段程序并没有像我们所期待的那样运行起来。
~\\  
\\为了解决这个问题,我们输入
\\\verb|ls -l|
\\来查看我们对这个文件的权限,我们会发现,\verb|My_name|的权限状态为
\\\verb|rw-rw-r--|
\\表明我们没有运行的权限。
~\\  
\\于是,我们输入
\\\verb|chmod 777 ~/My_name|
\\来更改权限状态。
~\\  
\\再次输入
~\\\verb|ls -l|
\\我们可以看到\verb|My_name|的权限状态为
\\\verb|rwxrwxrwx|。
~\\  
\\接下来,我们就可以通过
\\\verb|./My_name|
\\命令去尝试运行这个文件了。
~\\
~\\

\section{尝试不同的输入}
~\\
\\通过观察程序我们可以看到,程序中出现了\verb|$* $1|两个变量。
\\根据课本前面的介绍,我们知道\verb|$*|表示\verb|./My_name|的所有形参,\verb|$1|表示\verb|./My_name|的第一个行参。
~\\
\\同理可知,\verb|$i|表示\verb|./My_name|的第\verb|i|个行参\verb|(i>0)|。
\\特别注意的是,当\verb|i=0|时,\verb|$i|表示\verb|./My_name|自身的名字,即\verb|./My_name|。\par
~\\
\subsection{输入输出样例一}
~\\
\\如果我们运行
\\\verb|./My_name Rick Neil|
\\此时我们输入了两个行参,所以\verb|$*|上的值为,运行结果如样例一所示。
\begin{verbatim}
$ ./My_name
Original parameter are
Is your name ?
\end{verbatim}
~\\

\subsection{输入输出样例二}
~\\
\\如果我们直接运行
\\\verb|./My_name Rick Neil|
\\此时我们输入了两个行参,所以\verb|$*|的值是\verb|Rick Neil|,\verb|$1|的值是\verb|Rick|运行结果如样例二所示。
\begin{verbatim}
$ ./My_name Rick Neil
Original parameter are Rick Neil
Is your name Rick ?
\end{verbatim}
~\\

\subsection{输入输出样例三}
~\\
\\如果我们直接运行
\\\verb|./My_name Rick|
\\此时我们只输入一个行参,所以\verb|$*|和\verb|$1|的值相同都为\verb|Rick|,运行结果如样例三所示。
\begin{verbatim}
$ ./My_name Rick
Original parameter are Rick
Is your name Rick ?
\end{verbatim}
~\\
~\\

\section{总结}
~\\
\\通过该\verb|shell|程序样例,我掌握了以下知识点:
~\\
\begin{itemize}
\item 掌握了如何用\verb|shell|编程实现输入和输出返回值;
\item 了解了\verb|$*,$i(i>=0)|变量的含义;
\item 能熟练运用\verb|chown,chmod|函数更改\verb|shell|程序文件的所有者和权限。
\item ...
\end{itemize}



\bibliographystyle{plain}
\bibliography{ZL.bib}
\end{document}
