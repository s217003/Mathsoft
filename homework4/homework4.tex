\documentclass{ctexart}

\usepackage{graphicx}
\usepackage{amsmath}
\usepackage{amssymb}

\title{作业三:Mandelbrot Set 的生成和探索}

\author{邵柯欣 \\ 专业:信息与计算科学 学号:3200103310}

\begin{document}
\maketitle

\section{摘要}

若一个复系数的二次多项式$z_{n+1}=z_n^2+c$,在给定$z_0=0$的前提下,开始递归计算,存在复数$c$,使得该方程无限次迭代后的结果能保持有界(即不发散),将满足上述条件的复数${c}$的集合视为一种特殊集。$Adrien Douady$为纪念数学家$Benoit Mandelbrot$而将该特殊集命名为$Mandelbrot set$。

$Mandelbrot set$图像可以通过选取不同的复数$c$,并判断其数列${z_n}$是否达到无穷大(实践中通常指其是否在预设迭代次数后离开预设的某个含0在内的有界邻域)来实现:将$c$的实部作为二维平面的横坐标,虚部作为二维平面纵坐标,然后根据数列${z_n}$是否超过主观设定阈值来给每个点染色,若$ c $在经过预设迭代次数之后没有超过阈值(注意:这里是将曼德布洛特集图像与其补集图像区分开来的标志)则染成特殊的颜色(通常是黑色)。

因为$Mandelbrot set$图像在美学上拥有独特的吸引力,并且还是一个根据简单规则产生复杂结构的代表例子,这使得曼德布洛特集在数学之外的其他领域中也十分流行。它也是数学可视化和表现数学之美的最著名例子之一。 

\section{引言}
$Mandelbrot set$起源于20世纪初由法国数学家$Pierre Fatou$和$Gaston Julia$首先研究的复杂动力学。首次确切定义分形,并绘制出可视化的分形图案得益于$Robert W. Brooks$和$Peter Matelski$在1978年对$Kleinian Groups$的部分研究工作。在此基础上,1980年3月1日,在位于纽约的$Yorktown Heights$的$IBM$的$Thomas J. Watson Research Center$,$Benoît B. Mandelbrot$首次绘制出曼德布洛特集的可视化图形。且$Benoît B. Mandelbrot$在1980年发表了一篇关于二次多项式的参数空间$Parameter Space$的研究论文。对曼德布洛特集的数学研究真正始于数学家$Adrien Douady$和$John H. Hubbard$的一系列研究工作。他们探明了曼德布洛特集的许多基本性质,$Adrien Douady$ 为纪念$Benoît B. Mandelbrot$在分形几何中做出的杰出贡献,将该集合命名为$Mandelbrot set$。 \footnote{https://wiki.swarma.org/index.php/%E6%9B%BC%E5%BE%B7%E5%B8%83%E6%B4%9B%E7%89%B9%E9%9B%86_Mandelbrot_set}
\section{问题的背景介绍}
因为$Mandelbrot set$图像在美学上拥有独特的吸引力,并且还是一个根据简单规则产生复杂结构的代表例子,这使得曼德布洛特集在数学之外的其他领域中也十分流行。它也是数学可视化和表现数学之美的最著名例子之一。
随着个人计算机的普及,绘制$Mandelbrot set$图像变得十分方便快捷。
理论上$Mandelbrot set$图像可以无限放大,但实际操作时,由于$double$类型精度的影响,放大到一定程度会造成失真。本文中,将对$Mandelbrot set$图像进行缩放展示,探究不同部位$Mandelbrot set$图像的不同。
\section{数学理论}

$Mandelbrot set$指复数集${c}$,对任意的$c\in C$,给定$z_0=0$,满足经过方程$z_{n+1}=z_n^2+c$保持有界(即不发散)。

存在定理:复数趋向于无穷大,当且仅当它在某一点上具有模量>2。(仅适用于$z_{n+1}=z_n^2+c$)

所以,在用$C++$绘制$Mandelbrot set$图像时,我们只需要验证在给定的迭代次数里,$z_n$的模长是否大于2,就可以判断该二维平面的点所表示的复数$c$是否属于$Mandelbrot set$。

\section{算法}

\begin{verbatim}
dim z_0=0;
dim N=给定值;
给定a,b,d;
dim image=[a-d,a+d]*[b-d,b+d];
for p=(x,y) in image:
  the color of p = white;
  c=x+y*i;
  dim n=0;
  while n<N and |z_n|<=2:
    z_{n+1}=z_n^2+c;
    n=n+1;
  loop;
  if n=N:
    the color of p = black;
  endif
endfor
\end{verbatim}

\section{数值算例}
\section{结论}
\end{document}

1.摘要字数
在写论文摘要的时候,字数不需要太多,大家要把握一个原则:精简,能一句话讲完的事,大家不要用2句或3句来讲述,这样不仅占字数,还会显得你很啰嗦。
另外,如果是中文摘要的话,一般300个字以内就够了,英文摘要的话,则在250个实词以内即可。还需要注意的是,摘要中,不能出现图、表、化学结构式等东西哦~

2.摘要四部分

摘要主要有四部分,分别是研究背景/研究目的、研究内容/研究对象、研究方法、以及研究结果。其中,每一部分都很重要,大家要慎重对待每一部分。
二、如何写论文摘要

看完上面的内容,相信大家对摘要都有了一定的了解,那么论文摘要具体要怎么写呢?别急,下面就来分享两种较为实用的写法。

1.万金油法

万金油法,这种方法也被叫做捋大纲法,就是按照论文的大纲/目录,将论文主要研究的内容进行梳理、概括。基本什么类型或专业的论文都适用。基本句型:本文对XX进行了分析,采取了XX的研究方法,最终得到的XX结果。

2.提问法

提问法,简单来讲,就是主要回答四个问题,并将这四个问题的答案,进行汇总串联,最终形成一篇完整的论文摘要。

其中,主要回答这四个问题:

①本篇论文主要解决什么问题?

②本篇论文主要采用了什么研究方法?

③解决问题后得到什么结果?

④所得到的结果对实际有什么影响?

根据论文内容,回答上面这四个问题,再完成汇总即可。
