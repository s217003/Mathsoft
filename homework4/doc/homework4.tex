\documentclass{ctexart}

\usepackage{graphicx}
\usepackage{amsmath}
\usepackage{amssymb}

\title{作业三:Mandelbrot Set 的生成和探索}

\author{邵柯欣 \\ 专业:信息与计算科学 学号:3200103310}

\begin{document}
\maketitle

\section{摘要}

让一个复系数的二次多项式$z_{n+1}=z_n^2+c$,经过无穷次迭代依旧可以保持收敛的复数${c}$集合称为$Mandelbrot set$。本文将通过$C++$编程绘制,$Mandelbrot set$二维平面(x轴代表实部,y轴代表虚部)表示的图像(称为$Mandelbrot set$图像),并对该图像不同的位置进行缩放对比,切实感受$Mandelbrot set$图像的魅力。

\section{引言}

因为$Mandelbrot set$图像在美学上拥有独特的吸引力,并且还是一个根据简单规则产生复杂结构的代表例子,这使得曼德布洛特集在数学之外的其他领域中也十分流行。它也是数学可视化和表现数学之美的最著名例子之一。随着个人计算机的普及,绘制$Mandelbrot set$图像变得十分方便快捷。理论上$Mandelbrot set$图像可以无限放大,但实际操作时,由于$double$类型精度的影响,放大到一定程度会造成失真。本文中,将对$Mandelbrot set$图像进行缩放展示,探究不同部位$Mandelbrot set$图像的不同。

\section{问题的背景介绍}

$Mandelbrot set$起源于20世纪初由法国数学家$Pierre Fatou$和$Gaston Julia$首先研究的复杂动力学。首次确切定义分形,并绘制出可视化的分形图案得益于$Robert W. Brooks$和$Peter Matelski$在1978年对$Kleinian Groups$的部分研究工作。在此基础上,1980年3月1日,在位于纽约的$Yorktown Heights$的$IBM$的$Thomas J. Watson Research Center$,$Benoît B. Mandelbrot$首次绘制出曼德布洛特集的可视化图形。且$Benoît B. Mandelbrot$在1980年发表了一篇关于二次多项式的参数空间$Parameter Space$的研究论文。对曼德布洛特集的数学研究真正始于数学家$Adrien Douady$和$John H. Hubbard$的一系列研究工作。他们探明了曼德布洛特集的许多基本性质,$Adrien Douady$ 为纪念$Benoît B. Mandelbrot$在分形几何中做出的杰出贡献,将该集合命名为$Mandelbrot set$。

\section{数学理论}

$Mandelbrot set$指复数集${c}$,对任意的$c\in C$,给定$z_0=0$,满足经过方程$z_{n+1}=z_n^2+c$保持有界(即不发散)。\par

存在定理:复数趋向于无穷大,当且仅当它在某一点上具有模量>2。(仅适用于$z_{n+1}=z_n^2+c$)\par

所以,在用$C++$绘制$Mandelbrot set$图像时,我们只需要验证在给定的迭代次数里,$z_n$的模长是否大于2,就可以判断该二维平面的点所表示的复数$c$是否属于$Mandelbrot set$。\par

\section{算法}

$Mandelbrot set$图像可以通过在屏幕上选取不同点$p(x,y)$(代表复数$c$),并判断其所对应的数列${z_n}$是否在预设迭代次数内模长超过2,若未超过则对其进行染色(通常是黑色)来得到。

\begin{verbatim}
dim z_0=0;
dim N=给定值;
给定ox,oy,d;
dim image=[ox-d,ox+d]*[oy-d,oy+d];
for p=(x,y) in image:
  the color of p = white;
  c=x+y*i;
  dim n=0;
  while n<N and |z_n|<=2:
    z_{n+1}=z_n^2+c;
    n=n+1;
  loop;
  if n=N:
    the color of p = black;
  endif
endfor
\end{verbatim}

\section{数值算例}
\subsection{算例一:图片参数$ox=0.0,oy=0.0,d=2.5$}

\begin{figure}[h]
  \centering
  \includegraphics[width=0.8\textwidth]{../example/f1.bmp}
  \caption{整体的$Mandelbrot set$图像}
  \label{g::01}
\end{figure}

\subsection{算例一:图片参数$ox=+-1.0,oy=+-1.0,d=2.5$}

\begin{figure}[h]
  \centering
  \includegraphics[width=0.8\textwidth]{../example/f2.bmp}
  \caption{$Mandelbrot set$的局部放大}
  \label{g::02}
\end{figure}

\section{结论}
\end{document}
